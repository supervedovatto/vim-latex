<+	+>	!comp!	!exe!
%        File: !comp!expand("%:p:t")!comp!
%     Created: !comp!strftime("%a %b %d %I:00 %p %Y ").substitute(strftime('%Z'), '\<\(\w\)\(\w*\)\>\(\W\|$\)', '\1', 'g')!comp!
% Last Change: !comp!strftime("%a %b %d %I:00 %p %Y ").substitute(strftime('%Z'), '\<\(\w\)\(\w*\)\>\(\W\|$\)', '\1', 'g')!comp!
%
\documentclass[12pt]{amsart}
\usepackage{geometry}
\geometry{
A4paper,
left=20mm,
top=20mm,
right=20mm,
bottom=20mm
}
\usepackage[english]{babel}
%This package deals with input encodings. It provides a wider range of input encodings using standard mappings, than does inputenc; it also covers nearly all slots. In this way, it serves as more uptodate replacement for package inputenc.
\usepackage[utf8]{inputenx}
%This package can disable all hyphenation or enable hyphenation of non-alphabetics or monospaced fonts. The package can also enable hyphenation within ‘words’ that contain non-alphabetic characters (e.g., that include underscores), and hyphenation of text type­set in monospaced (e.g., cmtt) fonts.
\usepackage{hyphenat}
%The package typesets fractions “nicely” — in the form ‘a/b’ (i.e., staggered with a slash between them, rather than directly one over the other). The package is distributed as part of a bundle including the units package. Nicefrac’s facilities are provided, in a cleaner way, by the (experimental) xfrac package, but see also the faktor package for quotient spaces and the like.
\usepackage{units}
%Provides support for setting the spacing between lines in a document. Package options include singlespacing, onehalfspacing, and doublespacing. Alternatively the spacing can be changed as required with the \singlespacing, \onehalfspacing, and \doublespacing commands. Other size spacings also available.
\usepackage{setspace}
%Ex­ten­sive sup­port for hy­per­text in LATEX
\usepackage{hyperref}
%The pack­age adds PDF sup­port to the land­scape en­vi­ron­ment of pack­age lscape, by set­ting the PDF /Ro­tate page at­tribute. Pages with this at­tribute will be dis­played in land­scape ori­en­ta­tion by con­form­ing PDF view­ers.
\usepackage{pdflscape}
 % % % % % % % % Citações e referências % % % % % % % % % % % % % % % % % %
\usepackage[
citestyle=authoryear,
bibstyle=authoryear,
%dashed=false,
backend=biber,			%Specifies the database backend.
backrefstyle=all+,		%This option controls how sequences of consecutive pages in the list of back references are formatted.	
%backref=true,			%Whether or not to print back references in the bibliography
hyperref=true,			%Whether or not to transform citations and back references into clickable hyperlinks.
autopunct=true,			%This option controls whether the citation commands scan ahead for punctuation marks.
sorting=nyvt, 			%The sorting order of the bibliography
sortcites=true, 		%Whether or not to sort citations if multiple entry keys are passed to a citation command.
maxcitenames=3,			%A limit affecting all lists of names (author, editor, etc.).
abbreviate=true,		%Whether or not to use long or abbreviated strings in citations and in the bibliography.
arxiv=abs,				%Path selector for arXiv links.
isbn=false,				%This option controls whether the fields isbn/issn/isrn are printed.
url=false,				%This option controls whether the url field and the access date is printed. 
doi=false,				%This option controls whether the field doi is printed.
eprint=false,			%This option controls whether eprint information is printed.
firstinits=true,		%When enabled, all first and middle names will be rendered as initials. 
terseinits=false		%This option controls the format of initials generated by Biblatex.
]{biblatex}
% % % % % % % % % % % % % % % % % % % % % % % % % % % % % % % % % % % % % %
\addbibresource{/home/supervedovatto/Documents/Artigos/bibliografiageral.bib}
%\addbibresource{bibliografiageral.bib}
\renewcommand*{\revsdnamepunct}{} % Removing the comma after last name in bibliography 
\renewcommand*{\finalnamedelim}{\addspace\bibstring{and}\space} % Removes comma after penultimate name
%\renewcommand*{\revsdnamepunct}{} % Removing the periods after first name intial in bibliography
%\renewcommand{\labelnamepunct}{\addspace} % Removes period after year in bibliography
%\DeclareFieldFormat[article]{title}{#1}  % Remove quotes from journal title in bibliography
\DeclareNameAlias{sortname}{last-first} % Lists all authors with last name first in bibliography
% \DefineBibliographyStrings{portuguese}{%
% 	andothers = {\em \textit{et}\addabbrvspace \textit{al}\adddot}
% }

\title{<++>}
\author[VedoVatto\,T.]{Thiago VedoVatto}
\address[Thiago VedoVatto]{Instituto Federal de Educa\c{c}\~ao, Ci\^encia e Tecnologia de Goi\'as}
\email{thiago.vedovatto@ifg.edu.br}

\keywords{<++>}

\date{\today}

\maketitle

\begin{abstract}
<++>
\end{abstract}

<++>

\printbibliography

\newpage

\appendix

\end{document}
